\documentclass[a4paper, 10pt]{article}%тип документа

%Русский язык
\usepackage[T2A]{fontenc} %кодировка
\usepackage[utf8]{inputenc} %кодировка исходного кода
\usepackage[english,russian]{babel} %локализация и переносы

%отступы 
\usepackage[left=2cm,right=2cm,top=2cm,bottom=3cm,bindingoffset=0cm]{geometry}

%Вставка картинок
\usepackage{graphicx}
\graphicspath{}
\DeclareGraphicsExtensions{.pdf,.png,.jpg, .jpeg}

%Таблицы
\usepackage[table,xcdraw]{xcolor}
\usepackage{booktabs}

%Графики
\usepackage{pgfplots}
\pgfplotsset{compat=1.9}

%Математика
\usepackage{amsmath, amsfonts, amssymb, amsthm, mathtools}

%Заголовок
\author{Нугманов Булат \\ Подлесный Артём \\ группа 827}
\title{Работа 1.4.8 \\ Измерение модуля Юнга методом 
акустического резонанса}

\begin{document}
\maketitle
\section*{Цель работы}
Исследование явления акустического резонанса. Измерение скорости
распространения продольных колебаний в тонких стрежнях. Измерение модуля Юнга различных материалов. 
\section*{Оборудование}
Генератор звуковых частот, частотомер, осциллограф,
электромагнитные излучатель и приемник колебаний, набор стержней из различных материалов (стали, алюминия, меди).
\section*{Отчёт о работе}
\subsection*{Общая теория}
\subsubsection*{Распространение продольных волн в тонких стержнях}
Распространение продольных волн в тонких стержнях 
Акустические волны, распространяющиеся в металлических стержнях,
существенно отличаются от волн в неограниченной среде. Строгий анализ
распространения таких волн связан с довольно громоздкими математическими
расчетами. Будем рассматривать волны, длина $\lambda$ которых велика по
сравнению с радиусом $R$ стержня. Опишем распространение продольной
волны вдоль оси тонкого стержня постоянного сечения площадью $S$ .
Стержень считается тонким в том случае, когда радиус стержня $R$ мал по
сравнению с длиной волны $\lambda$, т.е. $R/\lambda\ll 1$.

Направим ось x вдоль геометрической оси стержня (рис. 1).
%вставь рис 1 из лабы 1.4.8
Под действием
продольной силы $F$ элементарный отрезок стержня $\Delta x$ , ограниченный
плоскостями $\Delta x$ и $(x+\Delta x)$, растянется или сожмется на величину $\Delta\xi=\dfrac{\partial\xi}{\partial x}\Delta x$, где $\dfrac{\partial\xi}{\partial x}$ — относительное удлинение, т. е. деформация элемента стержня. Напряжение $\sigma$ (т. е. сила, действующая на единицу поперечного
сечения стержня) согласно закону Гука равно
\begin{equation}
\sigma=\frac{F}{S}=E\dfrac{\partial\xi}{\partial x}
\end{equation}
Коэффициент пропорциональности $E$ носит название модуля Юнга и имеет размерность Н/м2. В результате переменной деформации вдоль оси стержня будет распространяться
продольная волна. Действительно, в сечениях $x$ и $x+\Delta x$
напряжения будут различными, а их разность можно записать следующим
образом: 
\begin{equation}
\sigma(x+\Delta x)-\sigma(x)=\frac{1}{S}\dfrac{\partial\xi}{\partial x}\Delta x=\frac{\partial}{\partial x}\left(\frac{F}{S}\right)\Delta x.
\end{equation}
Эта разность напряжений вызовет движение элемента стержня массой
$m=S\rho\Delta x$ вдоль оси $x$ ($\rho$ — плотность материала стержня). Используя
соотношения (1) и (2), на основании второго закона Ньютона уравнение
движения этого элемента можно записать в виде: 
\begin{equation}
S\rho\Delta x\dfrac{\partial^2\xi}{\partial t^2}=SE\dfrac{\partial^2\xi}{\partial x^2}\Delta x
\end{equation}
Обозначив $E/\rho$ через $c^2_{\text{ст}}$, выражение (3) запишем в следующем виде: 
\begin{equation}
\dfrac{\partial^2\xi}{\partial t^2}=c^2_{\text{ст}}\dfrac{\partial^2\xi}{\partial x^2}
\end{equation}
Это уравнение носит название волнового  уравнения. Оно, в частности,
описывает распространение продольных волн в стержне. Общее решение
волнового уравнения можно представить в форме двух бегущих волн, распространяющихся в обе стороны вдоль оси $x$ со скоростью
$c^2_{\text{ст}}$ :
\begin{equation}
\xi(x,t)=f(c_{\text{ст}}t-x)+g(c_{\text{ст}}t+x),
\end{equation}
где $f$ и $g$ — произвольные функции (определяемые начальными и граничными
условиями).
Параметр $c_{\text{ст}}$ в выражениях (4) и (5) имеет смысл скорости распространения волны. В рассматриваемом нами случае $R/\lambda\rightarrow 0$ скорость распространения
упругой продольной волны стремится к величине
\begin{equation}
c_{\text{ст}}\approx\sqrt{\dfrac{E}{\rho}}.
\end{equation}
В данной работе исследуются именно такие волны. \\
Отметим, что в высокочастотном (т. е. коротковолновом) пределе
при $\lambda\ll R$ скорость акустических волн в стержне стремится к скорости продольных волн в неограниченной среде ($\mu$ — коэффициент Пуассона): 
\begin{equation}
c_{\text{i}}=\sqrt{\dfrac{E(1+\mu)}{\rho(1+\mu)(1-2\mu)}}.
\end{equation}
\subsubsection*{Собственные колебания стержня}
В случае гармонического возбуждения колебаний с частотой $f$ продольная волна в тонком стержне может быть представлена в виде суперпозиции двух бегущих навстречу друг другу синусоидальных волн:
\begin{equation}
\xi(x,t)=A_1\sin(\omega t-kx+\varphi_1)+A_2\sin(\omega t+kx+\varphi_1),
\end{equation}
где $\omega=2\pi f$ — циклическая частота, коэффициент $k=2\pi/\lambda$ называют волновым числом или пространственной частотой. Здесь первое слагаемое описывает волну, бегущую в положительном направлении по оси $x$ , второе
— в отрицательном. Скорость их распространения равна
\[c_{\text{ст}}=\omega/k\]
Несложно показать, что при отражении синусоидальной волны от свободного конца стержня, её фаза не изменяется. Тогда
\begin{equation}
L=n\frac{\lambda_{\text{n}}{2}
\end{equation}
Таким образом, на длине стержня должно укладываться целое число полуволн.
\subsubsection*{Измерение скорости распространения  
продольных волн в стержне}
Зная плотность материала и величину скорости $c_{\text{ст}}$ можно по формуле (6) вычислить модуль Юнга материала $E$ . Для определения скорости $c_{\text{ст}}$ в данной работе используется метод акустического резонанса. Это явление состоит в том, что при частотах гармонического возбуждения, совпадающих с собственными частотами колебаний стержня $f\approxf_{\text{n}}$ , резко
увеличивается амплитуда колебаний, при этом в стержне образуется стоячая волна.
В данной работе возбуждение колебаний происходит посредством воздействия
на торец стержня периодической силой, направленной вдоль его
оси. Зная номер гармоники $n$ и частоту $f_{\text{n}}$, на которой наблюдается резонансное
усиление амплитуды колебаний, вызванных периодическим воздействием
на торец стержня, можно рассчитать скорость распространения
продольных волн в стержне:
\begin{equation}
c_{\text{ст}}=f_{\text{n}}\lambda_{\text{n}}=\dfrac{2Lf_{\text{n}}}{n}.
\end{equation}
Таким образом, для того, чтобы измерить скорость c_{\text{ст}}, нужно измерить
частоты резонансных гармоник для различных $n$, и зная геометрические
размеры стержня, рассчитать скорость по формуле (15). Далее, по
формуле (6) можно рассчитать и модуль Юнга материала, из которого
изготовлен стержень. Этот метод определения модуля Юнга материала
является одним из самых точных. 

\end{document}














































