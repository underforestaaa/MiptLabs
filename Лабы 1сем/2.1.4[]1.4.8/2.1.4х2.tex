\documentclass[a4paper, 10pt]{article}%тип документа

%Русский язык
\usepackage[T2A]{fontenc} %кодировка
\usepackage[utf8]{inputenc} %кодировка исходного кода
\usepackage[english,russian]{babel} %локализация и переносы

%отступы 
\usepackage[left=2cm,right=2cm,top=2cm,bottom=3cm,bindingoffset=0cm]{geometry}

%Вставка картинок
\usepackage{graphicx}
\graphicspath{}
\DeclareGraphicsExtensions{.pdf,.png,.jpg, .jpeg}

%Таблицы
\usepackage[table,xcdraw]{xcolor}
\usepackage{booktabs}

%Графики
\usepackage{pgfplots}
\pgfplotsset{compat=1.9}

%Математика
\usepackage{amsmath, amsfonts, amssymb, amsthm, mathtools}

%Заголовок
\author{Нугманов Булат \\ Подлесный Артём \\ группа 827}
\title{Работа 1.4.8 \\ Измерение модуля Юнга методом 
акустического резонанса}

\begin{document}

\begin{table}[]
\begin{tabular}{|c|c|c|c|c|c|c|c|c|}
\hline
\rowcolor[HTML]{EFEFEF} 
$\Delta R$, $\Omega$ & 0,01  & 0,02 & 0,03 & 0,04 & 0,05  & 0,06  & 0,07  & 0,09  \\ \hline
$t$, $c$             & 6,33  & 15,1 & 23,2 & 30,2 & 40,64 & 47,36 & 57,42 & 76,89 \\ \hline
\rowcolor[HTML]{EFEFEF} 
                     & 0,11  & 0,12 & 0,13 & 0,14 & 0,15  & 0,16  & 0,17  & 0,20  \\ \hline
                     & 95,23 & 103  & 111  & 120  & 128,5 & 139,1 & 148,1 & 176,4 \\ \hline
\end{tabular}
\end{table}
Алюминий\\
\begin{table}[]
\begin{tabular}{|c|c|c|c|c|c|c|c|c|c|c|}
\hline
\rowcolor[HTML]{EFEFEF} 
$\Delta R$, $\Omega$ & 0,02   & 0,04 & 0,06 & 0,08 & 0,10  & 0,12  & 0,14  & 0,16  & 0,18  & 0,20  \\ \hline
$t$, $c$             & 15,39  & 34,6 & 56   & 78,8 & 103,3 & 127,2 & 154,7 & 177,9 & 205,1 & 231,3 \\ \hline
\rowcolor[HTML]{EFEFEF} 
$\Delta R$, $\Omega$ & 0,22   & 0,24 & 0,26 & 0,28 & 0,30  & 0,32  & 0,34  & 0,36  & 0,38  & 0,41  \\ \hline
$t$, $c$             & 257,89 & 285  & 313  & 342  & 369,6 & 396,6 & 426,6 & 455,1 & 484,5 & 529,2 \\ \hline
\end{tabular}
\end{table}
Латунь\\
\begin{table}[h]
\begin{tabular}{|c|c|c|c|c|c|c|c|c|c|}
\hline
\rowcolor[HTML]{EFEFEF} 
$\Delta R$, $\Omega$ & 0,02  & 0,04 & 0,06 & 0,08 & 0,10  & 0,12  & 0,14  & 0,16  & 0,18  \\ \hline
$t$, $c$             & 14,99 & 35   & 54,1 & 77,9 & 100,6 & 126,7 & 152,4 & 179,7 & 206,6 \\ \hline
\rowcolor[HTML]{EFEFEF} 
$\Delta R$, $\Omega$ & 0,20  & 0,22 & 0,24 & 0,26 & 0,28  & 0,30  & 0,32  & 0,34  & 0,36  \\ \hline
$t$, $c$             & 234,2 & 264  & 290  & 332  & 363,7 & 391   & 421,5 & 451,5 & 482   \\ \hline
\end{tabular}
\end{table}
Железо
\begin{table}[h!]
\begin{tabular}{|c|c|c|c|c|c|c|c|c|c|c|c|}
\hline
\rowcolor[HTML]{EFEFEF} 
$\Delta R$, $\Omega$ & 0,01   & 0,02 & 0,03 & 0,04 & 0,05  & 0,06  & 0,07  & 0,09  & 0,10  & 0,11  & 0,13  \\ \hline
$t$, $c$             & 5,84   & 13,8 & 25,7 & 37,2 & 49,62 & 62,15 & 75,12 & 101,4 & 114,9 & 127,9 & 154,7 \\ \hline
\rowcolor[HTML]{EFEFEF} 
$\Delta R$, $\Omega$ & 0,15   & 0,17 & 0,19 & 0,21 & 0,23  & 0,25  & 0,27  & 0,29  & 0,31  & 0,33  & 0,35  \\ \hline
$t$, $c$             & 183,62 & 211  & 239  & 267  & 298,5 & 325,1 & 354,4 & 383,4 & 413,4 & 443,7 & 473,7 \\ \hline
\end{tabular}
\end{table}













\end{document}